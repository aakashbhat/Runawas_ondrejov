% mnras_template.tex 
%
% LaTeX template for creating an MNRAS paper
%
% v3.0 released 14 May 2015
% (version numbers match those of mnras.cls)
%
% Copyright (C) Royal Astronomical Society 2015
% Authors:
% Keith T. Smith (Royal Astronomical Society)

% Change log
%
% v3.0 May 2015
%    Renamed to match the new package name
%    Version number matches mnras.cls
%    A few minor tweaks to wording
% v1.0 September 2013
%    Beta testing only - never publicly released
%    First version: a simple (ish) template for creating an MNRAS paper

%%%%%%%%%%%%%%%%%%%%%%%%%%%%%%%%%%%%%%%%%%%%%%%%%%
% Basic setup. Most papers should leave these options alone.
\documentclass[fleqn,usenatbib]{mnras}

% MNRAS is set in Times font. If you don't have this installed (most LaTeX
% installations will be fine) or prefer the old Computer Modern fonts, comment
% out the following line
\usepackage{newtxtext,newtxmath}
% Depending on your LaTeX fonts installation, you might get better results with one of these:
%\usepackage{mathptmx}
%\usepackage{txfonts}

% Use vector fonts, so it zooms properly in on-screen viewing software
% Don't change these lines unless you know what you are doing
\usepackage[T1]{fontenc}

% Allow "Thomas van Noord" and "Simon de Laguarde" and alike to be sorted by "N" and "L" etc. in the bibliography.
% Write the name in the bibliography as "\VAN{Noord}{Van}{van} Noord, Thomas"
\DeclareRobustCommand{\VAN}[3]{#2}
\let\VANthebibliography\thebibliography
\def\thebibliography{\DeclareRobustCommand{\VAN}[3]{##3}\VANthebibliography}


%%%%% AUTHORS - PLACE YOUR OWN PACKAGES HERE %%%%%

% Only include extra packages if you really need them. Common packages are:
\usepackage{graphicx}	% Including figure files
\usepackage{amsmath}	% Advanced maths commands
% \usepackage{amssymb}	% Extra maths symbols

%%%%%%%%%%%%%%%%%%%%%%%%%%%%%%%%%%%%%%%%%%%%%%%%%%

%%%%% AUTHORS - PLACE YOUR OWN COMMANDS HERE %%%%%

% Please keep new commands to a minimum, and use \newcommand not \def to avoid
% overwriting existing commands. Example:
%\newcommand{\pcm}{\,cm$^{-2}$}	% per cm-squared

%%%%%%%%%%%%%%%%%%%%%%%%%%%%%%%%%%%%%%%%%%%%%%%%%%

%%%%%%%%%%%%%%%%%%% TITLE PAGE %%%%%%%%%%%%%%%%%%%

% Title of the paper, and the short title which is used in the headers.
% Keep the title short and informative.
\title[Ondrejov]{Runaway study}

% The list of authors, and the short list which is used in the headers.
% If you need two or more lines of authors, add an extra line using \newauthor
\author[Aakash Bhat et al.]{
Aakash Bhat,$^{1,2}$\thanks{E-mail: aakashbhat7@gmail.com}
Matti Dorsch,$^{1}$
Stephan Geier$^{1}$
Ulrich Heber$^{2}$
\\
% List of institutions
$^{1}$University of Potsdam, Potsdam, Germany\\
$^{2}$Dr. Karl-Remeis Sternwarte, Bamberg
}

% These dates will be filled out by the publisher
\date{Accepted XXX. Received YYY; in original form ZZZ}

% Enter the current year, for the copyright statements etc.
\pubyear{2015}

% Don't change these lines
\begin{document}
\label{firstpage}
\pagerange{\pageref{firstpage}--\pageref{lastpage}}
\maketitle

% Abstract of the paper
\begin{abstract}
Shortlisting stars, observing them, and then finding their origins.
\end{abstract}

% Select between one and six entries from the list of approved keywords.
% Don't make up new ones.
\begin{keywords}
runaways
\end{keywords}

%%%%%%%%%%%%%%%%%%%%%%%%%%%%%%%%%%%%%%%%%%%%%%%%%%

%%%%%%%%%%%%%%%%% BODY OF PAPER %%%%%%%%%%%%%%%%%%

\section{Introduction}


\section{Selection Procedure}

The selection of stars was done using the OB catalog of \citet{2003AJ....125.2531R}. We shortlisted sources with right ascension $\alpha$ between $17:00:00-04:00:00$ and declination $\delta>+20:00:00$, and a magnitude brighter than 11. This list was then cross-matched with Gaia DR3 and parallax values with a better uncertainty than 20\% and RUWE $<1.4$ were kept. The tangential velocities $V_{t}=4.75*\sigma/\pi$ and the Galactocentric coordinates x,y,z were computed for all the stars using the galpy module (\cite{2015ApJS..216...29B}). The final list was created using the cut-off $V_{t}>30$ kms$^{-1}$ and $z>0.29$ kpc, where the latter is the approximate scale height of the disk.

\begin{table}

 \begin{center}
%\scriptsize
\renewcommand{\arraystretch}{1.3}
\hspace*{-1cm}
\begin{tabular}{c c c c c c} 
\hline
\hline
Name & Spectral Type&App G&Parallax &V$_{\text{tan}}$ & Z \\
 &&&mas&kms$^{-1}$&  (kpc)\\
\hline
HD 121800 & B&9.06&0.29 & $61.63\pm 7.26$ &\\
* alf Cam& BSG(O)&4.21&0.59& $58.50\pm 14.31$ &\\
EM* VES  835& B&11.42& 0.27& $45.62\pm 4.54$ &\\
BD+34 1058 &O&8.86 &0.45 & $41.55 \pm  2.87$&\\
BD+34 1156 & B&9.14& 0.57& $39.11 \pm  1.33$&\\
HD 243827 &B&10.45 & 0.30 & $37.46 \pm 2.15$&\\
BD+34 1161&OB&10.36& 0.31& $34.21 \pm 3.16$&\\
HD 35327 &F2&6.45& 0.37 & $33.35 \pm 2.64$ &\\
*3 Gem & BSG(O)&5.73 &0.38 & $33.15 \pm 5.50$&\\
HD 36665 & Be* & 8.08& 0.70 & $33.01 \pm 1.50$&\\
BD+34 1150 & Em* &9.49& 0.49& $32.52 \pm 1.37$ &\\
HD 57742 &B9&6.45& 4.30 & $32.25 \pm  0.26$& \\
BD+59 804&B2&10.15& 0.38 & $32.22 \pm  2.27$&$0.36 \pm 0.02$\\
HD 250830 & F0	&9.72&0.46& $32.01 \pm  1.27$&\\
HD 253072 & A5&	9.6&0.30 & $31.91 \pm  1.94$& \\
HD 44597 &09V&	8.96 &0.52 & $31.65 \pm 1.43$&\\
HD 242935B &YSO	&~10 &0.21 & $31.45 \pm  7.70$&\\
HD 248894&O8&9.2 & 0.32  &$31.29 \pm  1.66$&\\
HD 277680 &EB*	&8.88 &0.41&  $31.27\pm 2.37$& \\
HD  27742 &B &5.96 &5.81& $30.76 \pm 0.25 $&\\
HD 277276&B	&9.4 & 0.78&$30.06\pm 0.71$&\\
\hline
HD 93521 &O	&6.97 &0.78& $13.03\pm 1.53$& $1.14\pm 0.11$\\
HD 77770 &B	&7.4& 0.94 & $13.48\pm 0.82$& $0.73 \pm 0.04$\\
HD 232992 &A7&9.2& 0.20 & $16.08\pm 1.20$& $0.50\pm 0.03$\\
HD 256959 &A&9.52& 0.26 & $18.52\pm 1.71$& $0.45 \pm 0.04$\\
BD+39 1328&BSG&9.64 &  0.17 & $24.37  \pm    2.72$ & $0.40 \pm 0.04$\\
HD 278115&B	&10.02&  0.23 &  $7.16  \pm    0.84$ & $0.37 \pm 0.03$\\
HD 278199&B	&9.35&  0.23 &  $3.87   \pm   0.84$ & $ 0.34 \pm 0.03$\\
HD 261071&B	&9.28 &  0.44 & $29.84  \pm    2.62$ & $0.33  \pm 0.03$\\
HD  50767&B	&7.72 &  0.67 & $13.56  \pm    1.33$ & $0.32 \pm 0.03$\\
V* IY Aur &&   &  0.52 & $22.98 \pm     1.18$ & $0.28  \pm 0.01$\\
HD 259494 &&   &  0.50 & $10.31   \pm   0.57$ & $0.27 \pm 0.01$\\
*  15 CVn  &&  &  3.92 & $17.32  \pm    0.26$ & $0.27 \pm 0.00$\\
HD 237211  &&  &  0.23 &  $5.43 \pm     0.58$ & $0.25 \pm 0.02$\\

\hline
\end{tabular}
\caption{Selected stars}
 \label{tab:significance}
\end{center}
\end{table}
\section{Conclusions}



\section*{Acknowledgements}




% The best way to enter references is to use BibTeX:

\bibliographystyle{mnras}
\bibliography{examples} % if your bibtex file is called example.bib


% Alternatively you could enter them by hand, like this:
% This method is tedious and prone to error if you have lots of references
%\begin{thebibliography}{99}
%\bibitem[\protect\citeauthoryear{Author}{2012}]{Author2012}
%Author A.~N., 2013, Journal of Improbable Astronomy, 1, 1
%\bibitem[\protect\citeauthoryear{Others}{2013}]{Others2013}
%Others S., 2012, Journal of Interesting Stuff, 17, 198
%\end{thebibliography}

%%%%%%%%%%%%%%%%%%%%%%%%%%%%%%%%%%%%%%%%%%%%%%%%%%

%%%%%%%%%%%%%%%%% APPENDICES %%%%%%%%%%%%%%%%%%%%%

\appendix

%%%%%%%%%%%%%%%%%%%%%%%%%%%%%%%%%%%%%%%%%%%%%%%%%%


% Don't change these lines
\bsp	% typesetting comment
\label{lastpage}
\end{document}

% End of mnras_template.tex
